\documentclass[12pt,letterpaper]{article}
\usepackage[top=.5in, bottom=1in, left=.5in, right=.5in]{geometry}
\usepackage{amsmath}
\usepackage{comment}
\usepackage{amssymb}
% These are Dan's custom extensions to try to make homework layout easier...
%% Submitter block
\newcommand{\studentblock}[5]{
	\begin{flushright}
	\parbox[t]{1.5in} {
		#1\\
		#2\\
		#3\\
		#4
	}
	\end{flushright}
}

%% Section headers
\newcommand{\hwSect}[1] {\noindent\large\bf#1\rm\normalsize}
\newcommand{\hwSubSect}[1]{\noindent\large#1\normalsize\newline}
\newcommand{\done}{$\hfill\square$}
\newcommand{\lname}[1]{\mathop{\kern0pt\mathit{#1}}}
\newcommand{\ifst}[3]{\text{ if } #1 \text{ then } #2 \text{ else } #3}
\newcommand{\lets}[2]{\text{ let }x=#1\text{ in }#2}
\newcommand{\true}{\text{True}}
\newcommand{\false}{\text{False}}
\newcommand{\bool}{\text{bool}}

\newenvironment{lambdac}
  {\catcode` =12 \setupspace
   \makeordinary{:}% to make the colon an ordinary symbol
   %%% possible other \makeordinary declarations
   $\mathgroup0 }
  {$}
{\catcode` =\active\gdef\setupspace{\def {\;}}}
\makeatletter
\newcommand{\makeordinary}[1]{\@tempcnta=\mathcode`#1
  \@tempcntb=\@tempcnta
  \divide\@tempcntb by "1000
  \multiply\@tempcntb by "1000
  \advance\@tempcnta by -\@tempcntb
  \mathcode`#1=\@tempcnta}
\makeatother

\title{CIS624 Homework 3}
\author{Adam Bates, Dan Ellsworth, Joe Pletcher}

\begin{document}
\maketitle
\noindent

\hwSect{Question 1}\\

\hwSubSect{1.1}
Let $fix=\lambda f.(\lambda x.f (x \; x)) (\lambda x.f (x \; x))$ and $fact=\lambda f.\lambda n. iszero(n) \; succ(n) \; (times \; n \; (f \; pred(n)))$ in $fix \; fact (\lambda s.\lambda z.s \; (s \; z))$ where $iszero$, $succ$, $times$, and $pred$ have the standard representations discussed in class.\\ \\
\noindent
This program is syntactically valid however it is non-terminating in pass-by-value due to the behavior of the $fix$ function.\\

\hwSubSect{1.2}
$\alpha$-conversion effectively provides static scope to the variables; which complicates generating a terminating case with different results. The insight that an abstraction is a value and cannot be further reduced until application was critical when devising this solution.\\ \\
\noindent
Consider the program $(\lambda a.\lambda b.a \; b) ((\lambda x.x) (\lambda y.y))$. Using pass-by-name, this program returns $\lambda b.((\lambda x.x) (\lambda y.y)) b$. Using pass-by-value this program returns $\lambda b.(\lambda y.y) b$.\\ \\
\noindent
Call-by-name and call-by-value can both be used to implement any computation even though some implmentation details will be different.\\ 


\hwSect{Question 2}

%% Static scoping means we need some kind of rule to rename things...

Syntax
\begin{eqnarray*}
M & ::= & x \\
&|& \lambda x.M \\
&|& M_1 M_2 \\
&|& n \\
&|& \true \\
&|& \false \\
&|& \ifst{M_0}{M_1}{M_2} \\
&|& \lets{M_1}{M_2} \\
\end{eqnarray*}

\begin{eqnarray*}
v & ::= & \lambda x.M \\
&|& n \\
&|& \true \\
&|& \false \\
\end{eqnarray*}

Judgments - axioms
\begin{eqnarray*}
\overline{\dashv \lambda x.M \Downarrow v} \\
\overline{\dashv n \Downarrow v} \\
\overline{\dashv \true \Downarrow v} \\
\overline{\dashv \false  \Downarrow v} \\
\end{eqnarray*}

Judgment - variables
\begin{eqnarray*}
{{(x,M)\in\Gamma \qquad \Gamma \dashv M \Downarrow v}\over{x \Downarrow v}}
\end{eqnarray*}

Judgments - if
\begin{eqnarray*}
\frac{\Gamma \dashv M_0 \Downarrow \true \qquad \Gamma \dashv M_1 \Downarrow v}{\Gamma \dashv \ifst{M_0}{M_1}{M_2} \Downarrow v} \\
\frac{\Gamma \dashv M_0 \Downarrow \false \qquad \Gamma \dashv M_2 \Downarrow v}{\Gamma \dashv \ifst{M_0}{M_1}{M_2} \Downarrow v} \\
\end{eqnarray*}

{\it all of the above apply to all cases}\\ \\

Judgments - apply -- by name
\begin{eqnarray*}
\frac{\Gamma \dashv M_1 \Downarrow \lambda x.M \qquad \Gamma \dashv [x \mapsto M_2]M \Downarrow v}{\Gamma \dashv M_1 M_2 \Downarrow v}
\end{eqnarray*}

Judgments - apply -- by value
\begin{eqnarray*}
\frac{\Gamma \dashv M_1 \Downarrow \lambda x.M \qquad \Gamma \dashv M_2 \Downarrow v_0 \qquad\Gamma \dashv [x \mapsto v_0]M \Downarrow v}{\Gamma \dashv M_1 M_2 \Downarrow v}
\end{eqnarray*}

Judgements - let -- by name dynamic
\begin{eqnarray*}
\frac{\Gamma,(x,M_1) \dashv M_2 \Downarrow v}{\Gamma \dashv \lets{M_1}{M_2} \Downarrow v}
\end{eqnarray*}

Judgements - let -- by value dynamic
\begin{eqnarray*}
\frac{\Gamma \dashv M_1 \Downarrow v_0 \qquad \Gamma,(x,v_0) \dashv M_2 \Downarrow v}{\Gamma \dashv \lets{M_1}{M_2} \Downarrow v}
\end{eqnarray*}


Judgements - let -- by name static
\begin{eqnarray*}
\frac{x\not\in dom(\Gamma) \qquad \Gamma,(x,M_1) \dashv M_2 \Downarrow v}{\Gamma \dashv \lets{M_1}{M_2} \Downarrow v} \\
\frac{x\in dom(\Gamma) \qquad x'\not\in dom(\Gamma) \qquad \Gamma,(x',M_1) \dashv [x \mapsto x']M_2 \Downarrow v}{\Gamma \dashv \lets{M_1}{M_2} \Downarrow v}
\end{eqnarray*}

Judgements - let -- by value static
\begin{eqnarray*}
\frac{x\not\in dom(\Gamma) \qquad \Gamma \dashv M_1 \Downarrow v_0 \qquad \Gamma,(x,v_0) \dashv M_2 \Downarrow v}{\Gamma \dashv \lets{M_1}{M_2} \Downarrow v} \\
\frac{x\in dom(\Gamma) \qquad x'\not\in dom(\Gamma) \qquad \Gamma \dashv M_1 \Downarrow v_0 \qquad \Gamma,(x',v_0) \dashv [x \mapsto x']M_2 \Downarrow v}{\Gamma \dashv \lets{M_1}{M_2} \Downarrow v}
\end{eqnarray*}


\hwSect{Exercise 8.3.4}\\

\begin{itemize}
\item {\tt E-IfTrue}: We are given that \begin{lambdac}(t_2, t_3)\end{lambdac} are of type \begin{lambdac}(T, T)\end{lambdac}. We know by the typing judgement that \begin{lambdac}if true then t_2 else t_3 : T\end{lambdac} and we know that \begin{lambdac}if true then t_2 else t_3 : T \rightarrow t_2 : T\end{lambdac} by the evaluation judgment.  

\item {\tt E-IfFalse}: We are given that \begin{lambdac}(t_2, t_3)\end{lambdac} are of type \begin{lambdac}(T, T)\end{lambdac}.  We know by the typing judgement that \begin{lambdac}if false then t_2 else t_3 : T\end{lambdac} and we know that \begin{lambdac}if false then t_2 else t_3 : T \rightarrow t_3 : T\end{lambdac} by the evaluation judgment.  

\item {\tt E-If}: We are given that \begin{lambdac}(t_1, t_2, t_3)\end{lambdac} are of type \begin{lambdac}(bool, T, T)\end{lambdac}.  Let us assume that \begin{lambdac}t_1 : bool \rightarrow {t_1}' : bool\end{lambdac}. We know by the evaluation judgement that \begin{lambdac}if t_1 then t_2 else t_3 \rightarrow if t_1' then t_2 else t_3 \end{lambdac} and that \begin{lambdac}if {t_1}' : bool then t_2 else t_3 : T\end{lambdac} by the typing judgment.  

\item {\tt E-Succ}:  Given \begin{lambdac}t_1 : Nat\end{lambdac}.  If we assume \begin{lambdac}t_1 : Nat \rightarrow {t_1}' : Nat\end{lambdac}, then by the typing judgement for succ we know that \begin{lambdac}succ t_1 : Nat\end{lambdac}.  From this we can conclude that \begin{lambdac}succ t_1 : Nat \rightarrow succ {t_1}' : Nat\end{lambdac}.

\item {\tt E-Pred}:  Given \begin{lambdac}t_1 : Nat\end{lambdac}.  If we assume \begin{lambdac}t_1 : Nat \rightarrow {t_1}' : Nat\end{lambdac}, then by the typing judgement for pred we know that \begin{lambdac}pred t_1 : Nat\end{lambdac}.  From this we can conclude that \begin{lambdac}pred t_1 : Nat \rightarrow pred {t_1}' : Nat\end{lambdac}.

\item {\tt E-IsZero}:  Given \begin{lambdac}t_1 : Nat\end{lambdac}.  If we assume \begin{lambdac}t_1 : Nat \rightarrow {t_1}' : Nat\end{lambdac}, then by the typing judgement for IsZero we know that \begin{lambdac}IsZero t_1 : Nat\end{lambdac}.  From this we can conclude that \begin{lambdac}IsZero t_1 : Nat \rightarrow IsZero {t_1}' : Nat\end{lambdac}.

\item {\tt E-PredZero, E-PredSucc, E-IsZeroZero, E-IsZeroSucc}: No inductive step is required on these evaluation rules.  The theorem holds by the typing judgments.

\end{itemize}
\hwSect{Exercise 9.2.2}\\
Show (by drawing derivation trees) that the following terms have the indicated types:

\begin{enumerate}
\item \begin{lambdac}f: Bool\rightarrow{Bool} \vdash f (if false then true else false) : Bool \end{lambdac}
\item \begin{lambdac}f: Bool\rightarrow{Bool} \vdash \lambda{x}:Bool. f (if x then false else x) : Bool\rightarrow{Bool}\end{lambdac}\\
\end{enumerate}

\hwSubSect{1.1}

\begin{eqnarray*}
{
{{f : \bool \rightarrow \bool~\in~f : \bool \rightarrow \bool}
\over{f : \bool \rightarrow \bool \vdash f: \bool \rightarrow \bool}}
~~
{
	{
		{{}\over{\false : \bool}}
		~~~~~
		{{}\over{\true : \bool}}
	}
\over{\ifst{\false}{\true}{\false} : \bool}}
}
\over{f : \bool \rightarrow \bool \vdash f (\ifst{\false}{\true}{\false} ) : \bool}
\end{eqnarray*}

\hwSubSect{1.2}
\begin{eqnarray*}
{
	{
		{
		{
			{}
			\over{f : \bool~\rightarrow\bool~\in~\Gamma}
		}
		\over{\Gamma~\vdash~f  : \bool \rightarrow \bool}
		}
		~~
		{
		{
			{
			{x : \bool~\in~\Gamma}
			\over{\Gamma~\vdash~x : \bool}
			}
			~~
			{
			{}
			\over{\Gamma~\vdash~\false : \bool}
			}
			~~
			{
			{x : \bool~\in~\Gamma}
			\over{\Gamma~\vdash~x : \bool}
			}
		}
		\over{\Gamma~\vdash~(\ifst{x}{\false}{x}) : \bool}
		}
	}
	\over{\Gamma, x : \bool~\vdash~f (\ifst{x}{\false}{x}) : \bool}
}
\over{f : \bool \rightarrow \bool~\vdash~\lambda x : \bool . f (\ifst{x}{\false}{x}) : \bool \rightarrow \bool}
\end{eqnarray*}

\end{document}