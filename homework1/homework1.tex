\documentclass[12pt,letterpaper]{article}
\usepackage[top=.5in, bottom=1in, left=.5in, right=.5in]{geometry}
\usepackage{amsmath}
\usepackage{comment}
\usepackage{amssymb}
% These are Dan's custom extensions to try to make homework layout easier...
%% Submitter block
\newcommand{\studentblock}[5]{
	\begin{flushright}
	\parbox[t]{1.5in} {
		#1\\
		#2\\
		#3\\
		#4
	}
	\end{flushright}
}

%% Section headers
\newcommand{\hwSect}[1] {\noindent\large\bf#1\rm\normalsize}
\newcommand{\hwSubSect}[1]{\noindent\large#1\normalsize\newline}
\newcommand{\done}{$\hfill\square$}


\title{CIS624 Homework 1}
\author{Adam Bates, Dan Ellsworth, Joe Pletcher}

\begin{document}
\maketitle
\noindent

\hwSect{2.2.6}\\
Given the binary relation $R$ on a set $S$. Define the relation $R'$ as $R \cup \{(s,s) | s \in S \}$. Show that $R'$ is the reflexive closure of $R$.\\ \\
Since $R$ is a relation on $S$ and $\{(s,s) | s \in S \}$ is a relation on $S$, the union of $R$ and  $\{(s,s) | s \in S \}$ must also be a relation on $S$. By definition a reflexive relation on $S$ must contain all pairs $\{(s,s) | s \in S \}$, therefore $R'$ must be reflexive by construction.\\ \\
To be the reflexive closure, $R'$ must be the smallest relation for which $R \subseteq R'$ and $\{(s,s) | s \in S \} \subseteq R'$. Assume $R'$ is not the smallest such relation, then there must be an element $e \in R'$ that can be removed, generating $R''$. This cannot be the case since $e$ must be an element of $R$ or $e$ must be an element of $\{(s,s) | s \in S \}$ due to the construction of $R'$. Removal of $e$ would violate definition of the reflexive closure since $R \not\subseteq R''$ or $\{(s,s) | s \in S \} \not\subseteq R''$.\done\\


\begin{comment}
We need to show that A) all $(s_i,s_i)$ are there for all $s \in S$ to be reflexive.  We also need to show that this is the smallest possible set.  We can do this by showing that we have a reflexive set R'.  It will be the reflexive closure of R if it is the smallest possible set with the property.  So if we remove a single element it is either a component of R, a component of R ref, or both.  So if we removed it we would be broken.  

PROOF BY CONTRADICTION QED MFER
done
\end{comment}


\hwSect{2.2.7}\\
Given the binary relation $R$ on a set $S$. Define $R^+$ as 
\begin{eqnarray*}
R_0 &=& R \\
R_{i+1} &=& R_i \cup \{(s,u) | \text{ for some } t, (s,t) \in R_i \text{ and } (t,u) \in R_i\} \\
R^+ &=& \bigcup_i R_i
\end{eqnarray*}
Show that $R^+$ is the transitive closure of $R$.\\ \\
For a relation $R^+$ to be transitive, it must be the case that $(s,u) \in R^+$ for all $t \in S$ such that $(s,t) \in R^+ \text{ and } (t,u) \in R^+$. By the iterative construction given, a missing relation $(s,u)$ at step $i$ of the form $t \in S$ such that $(s,t) \in R^+ \text{ and } (t,u) \in R^+$ will be added at step $i+1$.\\ \\
To be the transitive closure, $R^+$ must be the smallest relation on $S$ such that $R \subseteq R^+$ and that the transitive property holds. Assume $R^+$ is not the smallest such relation, then there must be some element $e$ that can be removed from $R^+$ without violating $R \subseteq R^+$ or the transitive property. This cannot be the case since $e$ was part of the original relation $R$ from step $0$ of the construction or $e$ was added at some step $i$ during the construction. If $e$ was in $R$, removal violates $R \subseteq R^+$. If $e$ was added during step $i$, removal would violate the transitive property; there would be some $t \in S$ where $(s,t) \in R^+$ and $(t,u) \in R^+$ and $(s,u) \not\in R^+$.\done\\ \\

\begin{comment}
We need to show that R+ really is the transitive closure.  The transitive closure means that for all s,t,u in S, if (s,t) and (t,u) are in R then (s,u) is also in R.

At each iterative step, we only add elements that are applications of the transitive property.  If we remove an element then we are either removing from R or removing from R tran.

How can we be guaranteed that we have everything?  R+ is the union of all transitive edges that we can possible add. If there was a missing transitive edge then we are not yet at R+.

BTW we know that R+ terminates because i must be less than or equal to the cardinality of the set S.
\end{comment}


\hwSect{2.2.8}\\
Suppose $R$ is a binary relation on $S$ and $P$ is a predicate on $S$ that is preserved by $R$. Show that $P$ is also preserved by $R^*$.\\ \\
For $P$ to be preserved by $R$, if we have $P(s)$ we must also have $P(t)$ for all $(s,t) \in R$.  We know that $R \subset R^*$ by the definition of $R^*$, and by the problem definition we know that $P$ is preserved.   The reflexive closure contains relations of the form $(s,s) \in S$, and trivially $P(s) \implies P(s)$. The transitive closure $R^+$ is also contained by $R^*$, and has relations of the form $(s,u)$ where, for some $t \in S$, $(s,t) \text{ and } (t,u) \in R^+$. It must be the case that $P(s) \implies P(u)$ since, for some $(s,t),(t,u) \in R$, $P(s) \implies P(t)$ and $P(t) \implies P(u)$.\done\\ \\

\begin{comment}
There are 3 types of elements in our new set: R, R ref, and R trans.  If the pair is from R then it obsviously still preserves P.  If the element is from R ref then the preservation of P is trivial because If s R t such that P(s) and P(t).

For R trans, we can look at all left terms in the relation as set S and all right terms in the relation as set T.  By applying R trans iterally we are adding new mappings from S to T.  However, we are not adding new elements to S or T; we have already evaluated the predicates of both sets in R and confirmed that R preserved P.  
\end{comment}

\hwSect{3.5.18}\\
To force the evaluation order, we must force particular terms to be reduced to values before other terms are reduced. To make the requested change, we must first remove rules B-IfFalse and B-IfTrue. These will be replaced by the following 5 new rules.\\ \\

\begin{eqnarray}
{t_2 \Downarrow v_2}\over{\text{if $t_1$ then $t_2$ else $t_3$} \Downarrow \text{if $t_1$ then $v_2$ else $t_3$}} \\
{t_3 \Downarrow v_3}\over{\text{if $t_1$ then $v_2$ else $t_3$} \Downarrow \text{if $t_1$ then $v_2$ else $v_3$}} \\
{t_1 \Downarrow \text{true}}\over{\text{if $t_1$ then $v_2$ else $v_3$} \Downarrow v_2} \\
{t_1 \Downarrow \text{false}}\over{\text{if $t_1$ then $v_2$ else $v_3$} \Downarrow v_3}
\end{eqnarray}

\end{document}