\documentclass[12pt,letterpaper]{article}
\usepackage[top=.5in, bottom=1in, left=.5in, right=.5in]{geometry}
\usepackage{amsmath}
\usepackage{comment}
\usepackage{amssymb}
% These are Dan's custom extensions to try to make homework layout easier...
%% Submitter block
\newcommand{\studentblock}[5]{
	\begin{flushright}
	\parbox[t]{1.5in} {
		#1\\
		#2\\
		#3\\
		#4
	}
	\end{flushright}
}

%% Section headers
\newcommand{\hwSect}[1] {\noindent\large\bf#1\rm\normalsize}
\newcommand{\hwSubSect}[1]{\noindent\large#1\normalsize\newline}
\newcommand{\done}{$\hfill\square$}
\newcommand{\lname}[1]{\mathop{\kern0pt\mathit{#1}}}


\newenvironment{lambdac}
  {\catcode` =12 \setupspace
   \makeordinary{:}% to make the colon an ordinary symbol
   %%% possible other \makeordinary declarations
   $\mathgroup0 }
  {$}
{\catcode` =\active\gdef\setupspace{\def {\;}}}
\makeatletter
\newcommand{\makeordinary}[1]{\@tempcnta=\mathcode`#1
  \@tempcntb=\@tempcnta
  \divide\@tempcntb by "1000
  \multiply\@tempcntb by "1000
  \advance\@tempcnta by -\@tempcntb
  \mathcode`#1=\@tempcnta}
\makeatother

\title{CIS624 Homework 2}
\author{Adam Bates, Dan Ellsworth, Joe Pletcher}

\begin{document}
\maketitle
\noindent

\hwSect{Question 1}\\
{\it Consider a different definition of fix:}
\begin{center}
\begin{lambdac}
fix = \lambda f. (\lambda x. f (x x ))(\lambda x. f (x x ))
\end{lambdac}
\end{center}
{\it Explain why this definition does not work for a call-by-value language.}\\ \\
\noindent
We recall that the evaluation \begin{lambdac}\it fix A {\rightarrow}^{\ast} A fix A \end{lambdac} by the definition of the fix function.  This version of fix is a Y-combinator that generates 2 applications that can execute immediately.  In call-by-value, this will lead to immediate unrolling of the latter term, \begin{lambdac}\it fix A \end{lambdac}.  The termination condition exists in {\it A}, so this function must partially evaluate first for this definition of {\it fix} to work correctly (as it would in a call-by-need language).\\

\noindent
A partial evaluation:\\

\begin{enumerate}
\item \begin{lambdac} fix A\end{lambdac}
\item \begin{lambdac}\lambda f. (\lambda x. f (x x ))(\lambda x. f (x x )) A\end{lambdac}
\item \begin{lambdac}(\lambda x. A (x x))(\lambda x. A (x x))\end{lambdac}
\item \begin{lambdac}A ((\lambda x. A (x x) ) (\lambda x. A (x x)))\end{lambdac}
\item \begin{lambdac}A fix A\end{lambdac}
\item At this point, a call-by-value language will continue to unfold (fix A) indefinitely before applying it to term A.
\end{enumerate}

\pagebreak

\hwSect{Question 2}\\
{\it Relate through an example the "variable-capture" problem with dynamic scope.} \\ \\
The variable-capture problem with dynamic scope occurs when a free variable matches the name of a bound variable within a substitution.\\

\noindent
Consider the function that returns the first of two terms applied (\begin{lambdac}f=\lambda x.\lambda z.x\end{lambdac})
in \begin{lambdac}f z y\end{lambdac}. The static expectation is that \begin{lambdac}f z y {\rightarrow}^{\ast} z\end{lambdac}. With naive substitution this is not the case since \begin{lambdac}z\end{lambdac} becomes bound during the substitution.\\

\noindent
\begin{lambdac}f z y \rightarrow (\lambda x.\lambda z.x) z y \rightarrow [x \mapsto z](\lambda z.x) y \rightarrow (\lambda z.z) y \rightarrow [z \mapsto y](z) \rightarrow y\end{lambdac}\\

\noindent
To avoid this problem, converting to static scope, free variables must be renamed before substitution so that they do not conflict with the bindings.\\

\noindent
\begin{lambdac}f z y \rightarrow (\lambda x.\lambda z.x) z' y \rightarrow [x \mapsto z'](\lambda z.x) y \rightarrow (\lambda z.z') y \rightarrow [z \mapsto y](z') \rightarrow z'\end{lambdac}

\end{document}